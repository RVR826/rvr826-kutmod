\chapter{Felhasználói dokumentáció}
\label{ch:user}

Ebben a fejezetben bemutatom az általam készített modul használatát. Kitérek a fejlesztés előtti tervezés végeredményére, ismertetem a \emph{user story-kat} és a hozzájuk tartozó funkciókat. Bemutatom a modul API függvényeinek használatát, integrálását, építésének folyamatát és a vizualizációs szoftver használatát. Ezek mellett ismertetem a hardveres és szoftveres követelményeket is a program futtatásához.
\begin{note}
    A projekt során a klasszikus "felhasználó" szerepkör nem egyértelmű. Mivel a keretprogram sem mindennapi használatra szánt termék, ezért az itt bemutatott kompozit sem az. Csak a teljes keretprogramnak van felhasználói felülete az autóban, de az túlmutat a dolgozat témáján, ezért csak képernyőfotókat mutatok be róla. A végfelhasználó ezzel fog majd kommunikálni. A fejezetben "felhasználó" alatt azt értem, aki a program másik \emph{subsystem-ével} dolgozik, de szüksége van az általam elkészített modul működésére, viszont a programcsomag részletes ismeretére nem. Felhasználó lehet még az, aki a későbbiekben jogszerűen hozzáfér a kész forráskódhoz, de csak integrálni szeretné a saját kompatibilis útvonaltervezőjébe, és erre engedélye is van.
\end{note}
\clearpage

\section{A modul funkciói}
Ebben a részben ismertetem az elkészített \emph{Waypoint Module} funkcióit, és a felhasználói eseteket különböző felhasználóknak és a fejlesztőknek is. Először bemutatom a \emph{user story-kat}, majd részletesebben is kifejtem a felhasználói eseteket.

\subsection{\emph{User story-k}}

\begin{center}
	\begin{longtable}{ | p{0.5\textwidth} | p{0.5\textwidth} | }
		\hline
		\emph{User story} & Elfogadási kritérium
		\\ \hline \hline
        \endfirsthead
        
		\hline
        \emph{User story} & Elfogadási kritérium
		\\ \hline \hline
        \endhead
		
		\textbf{AS} a user, & \textbf{GIVEN} a scene fails, \\
        \textbf{I WANT} to be able to plan paths using waypoint generation, & \textbf{WHEN} I use the planner without waypoints, \\
        \textbf{SO THAT} more scenes can be completed & \textbf{THEN} I can try the scene with waypoint generation \\
        \hline
		
		\textbf{AS} a user, & \textbf{GIVEN} a scene passed before, \\
        \textbf{I WANT} to be able to disable waypoint generation, & \textbf{WHEN} I used the planner without waypoints, \\
        \textbf{SO THAT} faster runtime and shorter paths are achived & \textbf{THEN} I can disable the module from the parameters \\
        \hline
        
		\textbf{AS} a user, & \textbf{GIVEN} the source code is available, \\
        \textbf{I WANT} to build the module easily, & \textbf{WHEN} I build, \\
        \textbf{SO THAT} it is compatible with the planner library & \textbf{THEN} I can use  {\fontfamily{cmtt}\selectfont CMakeLists.txt} files for \emph{CMake}\\
        \hline
        
		\textbf{AS} a user, & \textbf{GIVEN} the module is on, \\
        \textbf{I WANT} to customize the values for RC generation, & \textbf{WHEN} I run a path planning, \\
        \textbf{SO THAT} it is compatible with my vehicle & \textbf{THEN} I can set every RC and cost parameter \\
        \hline
        
		\textbf{AS} a developer, & \textbf{GIVEN} I want to test the project, \\
        \textbf{I WANT} to have a separate test executable, & \textbf{WHEN} I run all tests, \\
        \textbf{SO THAT} I don't interfere with the other composite's tests and I can generate a testing report for my module & \textbf{THEN} I can see the new test section and it's tests passing \\
        \hline
        
		\textbf{AS} a developer, & \textbf{GIVEN} I open the visualization software, \\
        \textbf{I WANT} to have visualization integration for the module, & \textbf{WHEN} I want to switch between operating modes, \\
        \textbf{SO THAT} I can test and demo the changes between the two operating modes & \textbf{THEN} I can do so with a checkbox \\
        \hline
        
    	\caption{\emph{User story-k}}
    	\label{tab:userStories}
	\end{longtable}
\end{center}

\Az\told{\ref{tab:userStories}}+as{} táblázat mutatja a projekt \emph{user story-jait}. A \emph{Waypoint Module} működésének minimális elvárásait fogalmazza meg mindkét felhasználói csoport számára.

\subsection{Felhasználói esetek különböző felhasználóknak}
A fejezet elején két fő csoportot fogalmaztam meg a modul felhasználóinak, ezek röviden a következők:
\begin{itemize}
    \item A könyvtárt használó fejlesztő egy másik \emph{subsystem-ből}
    \item A könyvtárt egy másik projektbe intergráló fejlesztő
\end{itemize}

Mindkét csoportnak más lépések szükségesek a használathoz, de vannak közös követelmények is.
\begin{note}
    A közös követelmények főleg a futási környezetben és a programozási nyelvben értendők. Ezekről egy későbbi alpontban térek ki.
\end{note}

Az első csoportba tartozó felhasználóknak csak egy kisebb részét szükséges ismerni az egész modulnak ahhoz, hogy használni tudják a \emph{waypoint} generálást. Ha csak a körülötte lévő útvonaltervezővel együtt kívánja használni, akkor még kevesebb információ szükséges. Mivel a modul már integrálva van az útvonaltervezőbe, ezért a már meglévő paraméterekkel vezérelhető a működése. Itt annyi lehetséges, hogy ki- illetve bekapcsoljuk a \emph{waypoint-ok} használatát. Ennek a folyamata \az\told{\ref{src:toggleWaypoints}}+as{} kódrészleten látható.
\clearpage

\lstset{caption={A \emph{Waypoint Module} ki- és bekapcsolása, állapotának lekérdezése}, label=src:toggleWaypoints}
\begin{lstlisting}[language={C++}]
// Default parameters (waypoints disabled)
CPpPathPlannerParams params{CPpParams{}}; 

// Enable
params.overwrite<config::CEnableWaypointModuleParams(
    config::CEnableWaypointModuleParams::data_type{true}
);

// Disable
params.overwrite<config::CEnableWaypointModuleParams(
    config::CEnableWaypointModuleParams::data_type{false}
);

// Returns true if the module is enabled, false otherwise
const auto isEnabled = params.get<config::CEnableWaypointModuleParams>();
\end{lstlisting}

\begin{note}
    A modul hozzáadása interfészt nem tört, csak plusz funkcionalitást adott az útvonaltervezőhöz, így aki eddig használta, annak nem kell lényeges változtatást eszközölnie a kódbázisában.
\end{note}

Azok a felhasználók, akik integrálni szeretnék a saját rendszerükbe a modult, az osztályok API függvényeivel kell kommunikálniuk. Szükséges a könyvtár megfelelő használatához:
\begin{itemize}
    \item {\fontfamily{cmtt}\selectfont vfc} keretrendszer
    \item {\fontfamily{cmtt}\selectfont common::*} osztályok
\end{itemize}

Az útvonaltervező sem része a modulnak, így annak is rendelkezésre kell állnia. Mivel a \emph{waypoint-ok} egy diszkrét pont segítségével vannak számítva, ezért \emph{Motion Primitive-ekkel} és diszkrét pontokkal ábrázolt útvonalakon is használhatóak. Az integráció lépései a következők:
\begin{enumerate}
    \item Ha szükséges, a {\fontfamily{cmtt}\selectfont vfc} keretrendszer helyettesítése
    \item Ha szükséges, a {\fontfamily{cmtt}\selectfont common::*} osztályok helyettesítése
    \item A minimális elmozdulást leíró paraméterek finomítása a megfelelő járműhöz
    \item A költségszámításhoz használt paraméterek finomítása a megfelelő járműhöz
    \item A modul osztályainak tényleges integrációja a kódbázisba
\end{enumerate}

Mind a minimális elmozdulást leíró, mind a költségszámításhoz használt paraméterek a {\fontfamily{cmtt}\selectfont waypoint\_module\_params.hpp} osztályban találhatóak. Ezek beállítása fontos, hiszen ezen értékek alapján generálhatóak, illetve rendezhetőek sorba a köztes pontok.

A végfelhasználó az a személy, aki a teljes projektet használja a járműben. Ők közvetlenül nem használják a modult, hiszen egy kész, beépített terméket kapnak. A felhasználói felületen csak a teljes útvonal látható számukra, függetlenül annak előállításától. Egy képernyőkép a végfelhasználó által látható vezérlőről \az\told{\ref{fig:userInterface}}+as{} ábrán látható.
\begin{figure}[H]
    \centering
	\includegraphics[angle=180,width=1\textwidth]{user_interface}
	\caption{Az autós felhasználói felület}
	\label{fig:userInterface}
\end{figure}
\clearpage

\section{A program építése és a vizualizáció futtatása}

Ebben a részben bemutatom a \emph{Waypoint Module} és a vizualizációs alkalmazás építését, futtatását és különböző követelményeiket. Kitérek a szoftveres, hardveres előfeltételekre is.

\subsection{Futási környezet, hardveres és szoftveres követelmények}

A program forráskódját C++ nyelven készítettem el, a C++17-es verzió alapján. Mivel a nyelv lehetővé teszi a \emph{cross-platform} építést és futtatást, ezért az operációs rendszerre nincs sok követelmény. Bármilyen 64 bites operációs rendszeren futtatható, ahol natívan támogatva vannak a dupla pontosságú lebegőpontos számokkal végzett műveletek. A vizualizációs szoftver a Qt keretrendszer segítségével készült, így az is támogatja a \emph{cross-platform} funkciókat. Ezzel ellentétben mind a \emph{Waypoint Module}, mind a vizualizációs szoftver csak Linux rendszeren volt építve és futtatva a fejlesztési folyamatban.

Az általam készített modul teljes egésze a \emph{stack-en} fut, mivel a projekt keretein belül nem használható dinamikus memória. A beágyazott rendszerek (főleg járművek) esetén szükséges ismerni a program által használt memória mennyiségét, hiszen ez alapján tudják szabályozni a ráfordítandó erőforrásokat. A következő táblázatban található az útvonaltervező és azon belül a \emph{Waypoint Module stack size-a}.

\begin{table}[h]
    \centering
    \begin{tabular}{|c|c|}
        \hline
        \textbf{Könyvtár} & \textbf{Méret} \\
        \hline
        \hline
        
        Teljes útvonaltervező & 42.984.448 bájt (42,98 Mb) \\
        \hline
        Ebből a \emph{Waypoint Module} & 1712 bájt (1,71 Kb) \\
        \hline
    \end{tabular}
    \label{tab:stackSizes}
    \caption{Útvonaltervező \emph{stack size-a}}
\end{table}

\begin{note}
   A méretek változhatnak, ha az útvonaltervező, vagy bármelyik osztályának \emph{template} argumentumai változnak.
\end{note}

\subsection{A \emph{Waypoint Module} építése}
Mind az általam készített modul, mind a vizualizációs szoftver az építéshez a \emph{CMake build} rendszert használja. Ez viszont csak az alapját képezi az egész folyamatnak, hiszen minden \emph{subsystem-nek} és kompozitnak vannak függőségei. Ezen függőségek automatikus beszerzéséről és a megfelelő környezet kialakításáról a \emph{Dockerized Toolchain Interface} (később {\fontfamily{cmtt}\selectfont dti}) rendszer felelős. A projekt kontextusában a {\fontfamily{cmtt}\selectfont dti} egy elosztott architektúrát használ, amely több \emph{Docker} konténerre épül. Az ilyen megközelítés egyik előnye a felhasználók számára, hogy az adott feladathoz a megfelelő konténert tudják használni. A {\fontfamily{cmtt}\selectfont dti} API \emph{Gateway} egy egységes interfésszel biztosítja az összes konténerhez való elérést. A rendszer használatának előnyei:
\begin{itemize}
    \item \textbf{egységes interfész biztosítása a \emph{toolchain-hez}:} \\
    A felhasználónak elegendő csak a {\fontfamily{cmtt}\selectfont dti} eszközt ismernie (és a súgó funkció használatát), hogy az összes konténert használni tudja. Nincs szükség arra, hogy a felhasználó tudja, mely parancsokat melyik \emph{Docker} konténer hajtja végre, mivel ezt a rendszer automatikusan kezeli, és a parancsokat a megfelelő konténerbe továbbítja.

    \item \textbf{egyszerű bővíthetőség:} \\
    A felhasználók könnyen hozzáadhatnak új konténereket a rendszerhez anélkül, hogy új interfészt kellene készíteni és a használatát megtanulni, dokumentálni. Lehetőséget ad még ezek átlátható integrációjára a \emph{gateway} módosítása nélkül.
\end{itemize}

\begin{note}
    A {\fontfamily{cmtt}\selectfont dti} API \emph{Gateway} a telepítés után bármilyen \emph{bash} konzolból használható, csak internethozzáférés szükséges hozzá. Az utasítások felépítése a következő: {\fontfamily{cmtt}\selectfont dti <command> <options>}. A {\fontfamily{cmtt}\selectfont dti --help} segítségével nyitható meg a súgó.
\end{note}

\lstset{caption={Legfontosabb {\fontfamily{cmtt}\selectfont dti} utasítások}, label=src:buildCommands}
\begin{lstlisting}[language={bash}]
# build the project (or part) in release mode
dti build

# build the project (or part) in debug mode
dti build linux-x86_64-gcc-8-debug

# incremental build of the project,
# requires at least one normal build in it's respective build mode
dti build --build

# build the project and every dependency that cannot be found
dti build --build-packages="missing"

# run all unit tests in the current suite
dti test
\end{lstlisting}

\subsection{A vizualizációs szoftver építése és futtatása}
A vizualizációs szoftver építéséhez külön \emph{Docker} konténer áll rendelkezésre a {\fontfamily{cmtt}\selectfont dti}-on belül. Ezt a {\fontfamily{cmtt}\selectfont dti build-qt <options>} utasítás segítségével érhetjük el. Itt is lehetőség van a \emph{debug} és a \emph{release build-ek} elkülönítésére. A vizualizációs szoftver tekintetében ez különösen fontos, hiszen az elkészült implementációkat itt tudjuk futtatni, és \emph{debug} módban több helyen van lehetőség betekinteni az objektumok felépítésébe. \emph{Release} módban ez azért nem lehetséges, mert ebben a fordító magasszintű optimalizálással építi a szoftvert, ezért sok értéket kioptimalizálnak, és ezek számunkra elérhetetlenek.

Építés után a kész bináris állomány a \emph{./\_generated/build/linux-x86\_64-gcc-8/Debug/bin/qt\_visu} vagy a \emph{./\_generated/build/linux-x86\_64-gcc-8/Release/bin/qt\_visu} helyen található a \emph{build mode} függvényében. Ezt \emph{linux} rendszeren konzolból vagy akár közvetlenül a mappából is van lehetőség futtatni.

\begin{note}
    A fejlesztés alatt a \emph{Visual Studio Code} egyik bővítményét használtam a vizualizáció futtatására. A projektben adott volt az integrált futtatáshoz szükséges összes adat a {\fontfamily{cmtt}\selectfont launch.json}, {\fontfamily{cmtt}\selectfont settings.json} és {\fontfamily{cmtt}\selectfont tasks.json} állományokban. Ezek segítségével egy gombnyomással indítható a program, és beépített \emph{debug} ablak is a rendelkezésemre áll.
\end{note}

\section{A \emph{Waypoint Module} használata}

Ahogy azt egy korábbi pontban említettem, az egyik felhasználói csoportnak szüksége van a \emph{Waypoint Module} osztályainak az API függvényeinek ismeretére. Ezek azok az interfészfüggvények és metódusok, amelyek segítségével kommunikálni lehet az objektumokkal. Ebben a fejezetben minden osztály használatát röviden ismertetem.

\subsection{{\fontfamily{cmtt}\selectfont THeapElement}}

A {\fontfamily{cmtt}\selectfont THeapElement} osztály a \emph{heap} elemeit megvalósító \emph{template}, melynek argumentuma a benne tárolt adat típusa. Ezt egészíti ki egy költségértékkel, amely szerint sorba is rendezhetők az objektumok. Az API függvényei a következők:
\begin{itemize}
    \item {\fontfamily{cmtt}\selectfont getCost()} - lekérdezi a költségértéket
    \item {\fontfamily{cmtt}\selectfont setCost(vfc::float32\_t f\_cost)} - beállítja a költségértéket
    \item {\fontfamily{cmtt}\selectfont value()} - lekérdezi a tárolt adatértéket
\end{itemize}

\subsection{{\fontfamily{cmtt}\selectfont CMinBinaryHeap}}

A {\fontfamily{cmtt}\selectfont CMinBinaryHeap} osztály egy \emph{min binary heap-et} megvalósító osztály, melynek \emph{template} argumentumai a benne tárolt érték típusa, és a maximális méret. Nem kötelező az előző alpontban említett osztályt használni tárolt típusnak, de előfeltétel, hogy sorbarendezhetőek legyenek a benne lévő értékek. Az API függvényei a következők:
\begin{itemize}
    \item {\fontfamily{cmtt}\selectfont size()} - lekérdezi a tárolt elemek számát
    \item {\fontfamily{cmtt}\selectfont insert(const HeapElementType\& f\_heapElement)} - beszúr egy új elemet a \emph{heap-be}, ha az nincs tele
    \item {\fontfamily{cmtt}\selectfont removeMinimum()} - lekérdezi és törli a minimum költségű elemet
    \item {\fontfamily{cmtt}\selectfont getMinimum()} - lekérdezi a minimum költségű elemet
    \item {\fontfamily{cmtt}\selectfont clear()} - kiüríti a \emph{heap-et}
    \item {\fontfamily{cmtt}\selectfont isEmpty()} - lekérdezi, hogy üres-e a \emph{heap}
\end{itemize}

\subsection{{\fontfamily{cmtt}\selectfont CWaypointModuleParams}}
A {\fontfamily{cmtt}\selectfont CWaypointModuleParams} osztály felelős a \emph{waypoint} számításhoz, és a költségszámításhoz szükséges értékek tárolásáért. A benne lévő összes értékhez tartozik egy \emph{getter} és egy \emph{setter} függvény a megfelelő paraméterrel. Ezek az értékek a következők:
\begin{itemize}
    \item {\fontfamily{cmtt}\selectfont m\_minMoveLength} - Minimum mozgás egyenes irányba
    \item {\fontfamily{cmtt}\selectfont m\_sigmaParallelLongitudinal} - Párhuzamos mozdulat esetén a hosszanti elmozdulás
    \item {\fontfamily{cmtt}\selectfont m\_sigmaParallelLateral} - Párhuzamos mozdulat esetén a keresztirányú elmozdulás
    \item {\fontfamily{cmtt}\selectfont m\_sigmaOrthogonalLongitudinal} - Merőleges mozdulat esetén a hosszanti elmozdulás
    \item {\fontfamily{cmtt}\selectfont m\_sigmaOrthogonalLateral} - Merőleges mozdulat esetén a keresztirányú elmozdulás
    \item {\fontfamily{cmtt}\selectfont m\_sigmaYaw} - A jármű állása beli eltérés a mozdulat végére
    \item {\fontfamily{cmtt}\selectfont m\_paramMoveLength} - Költségszámításnál használt, az úthossz paramétere
    \item {\fontfamily{cmtt}\selectfont m\_paramDirectionChange} - Költségszámításnál használt, az irányváltások számának paramétere
    \item {\fontfamily{cmtt}\selectfont m\_paramDeviationX} - Költségszámításnál használt, az $x$-beli eltérés paramétere
    \item {\fontfamily{cmtt}\selectfont m\_paramDeviationY} - Költségszámításnál használt, az $y$-beli eltérés paramétere
    \item {\fontfamily{cmtt}\selectfont m\_paramDeviationYaw} - Költségszámításnál használt, az $\theta$-beli eltérés paramétere
\end{itemize}

\subsection{{\fontfamily{cmtt}\selectfont CReferenceConfigurationCalculator}}

A {\fontfamily{cmtt}\selectfont CReferenceConfigurationCalculator} osztály felelős a köztes pontokat képző \emph{reference configuration-ök} generálásáért egy adott \emph{landmark-hoz} és a célponthoz képest. Feladata még ezen pontok tárolása és sorbarendezése is. Az API függvényei a következők:
\begin{itemize}
    \item {\fontfamily{cmtt}\selectfont addLandmark(const common::CPose\& f\_landmark)} - új \emph{landmark} felvétele a tárolóba
    \item {\fontfamily{cmtt}\selectfont getNextReferenceConfigurations()} - a célponthoz képest legkedvezőbb két \emph{reference configuration} lekérdezése
    \item {\fontfamily{cmtt}\selectfont reInit(const common::CPose\& f\_startPose, const common::CPose\& f\_targetPose)} - az objektum újrainicializálása új start- és célponttal
    \item {\fontfamily{cmtt}\selectfont setWaypointModuleParams(const CWaypointModuleParams\& f\_params)} - a paraméterek frissítése
\end{itemize}

\begin{note}
    Az {\fontfamily{cmtt}\selectfont addLandmark} függvény további paraméterekkel kiegészíthető, ha ezek rendelkezésre állnak a hozzáadás pillanatában. A teljes paraméterlista a következő:
    \begin{compactitem}
        \item {\fontfamily{cmtt}\selectfont const common::CPose\& f\_landmark} - a hozzáadni kívánt \emph{landmark}
        \item {\fontfamily{cmtt}\selectfont common::Metre f\_pathLength} - az útvonal hossza az előző ponttól ebbe a pontba
        \item {\fontfamily{cmtt}\selectfont vfc::int32\_t f\_directionChangeCount} - az eredeti \emph{landmark-tól} az idáig vezető úton tett irányváltások száma
    \end{compactitem}
\end{note}

\subsection{{\fontfamily{cmtt}\selectfont CPathDatabase}}

A {\fontfamily{cmtt}\selectfont CPathDatabase} osztály nem közvetlenül a \emph{Waypoint Module} része, de a teljes útvonaltervező működéséhez elengedhetetlen. Ez egy nagyon projekt-specifikus megvalósítása a fa alapú tárolónak, hiszen nem csak a generált útvonalrészeket kell megtartania, hanem a változó hosszúságú összekötő mozdulatok is itt vannak példányosítva. Ezt a funkcionalitást dinamikus memóriahasználattal el lehet kerülni. Az osztály (nem implementációspecifikus) API függvényei a következők:
\begin{itemize}
    \item {\fontfamily{cmtt}\selectfont addPath(const common::CPosture\& f\_startPosture, const common::CPosture\& f\_targetPosture, const MotionPrimitiveResultVector\& f\_primitives)} - egy új útvonal beszúrása a tárolóba a start- és a célpont közötti primitívekkel megadva
    \item {\fontfamily{cmtt}\selectfont getDirectPathEndingWith(const common::CPosture\& f\_targetPosture)} - egy megadott ponttal végződő útvonalrészlet lekérdezése
    \item {\fontfamily{cmtt}\selectfont getFullPathFromStartToTarget(const common::CPosture\& f\_startPosture, const common::CPosture\& f\_targetPosture)} - a teljes útvonal lekérdezése a megadott start-, és célpont között
    \item {\fontfamily{cmtt}\selectfont clear()} - a tároló kiürítése
\end{itemize}

\subsection{A {\fontfamily{cmtt}\selectfont vfc} keretrendszer és {\fontfamily{cmtt}\selectfont common::*} osztályok}

Az előző pontban említettem, hogy a {\fontfamily{cmtt}\selectfont vfc} keretrendszer és {\fontfamily{cmtt}\selectfont common::*} osztályok szükségesek a \emph{Waypoint Module} működéséhez. Ha ezek nem állnak rendelkezésre, helyettesíteni kell őket más, hasonló funkcionalitású elemekkel.

A {\fontfamily{cmtt}\selectfont vfc} keretrendszer főleg elemi típusokat, tároló objektumokat, matematikai műveleteket és \emph{pointer-ekkel} végzett műveleteket valósít meg. Ennek a komponensnek a helyettesítésére megfelelő lehet az {\fontfamily{cmtt}\selectfont std} standard C++ könyvtár, hiszen ezekben megtalálható minden olyan osztály, amit az implementációban egy {\fontfamily{cmtt}\selectfont vfc}-beli osztály valósít meg. Nem helyettesíti viszont az erősen típusossághoz szükséges mértékegységeket, amik fontos szerepet töltenek be a teljes projekt helyes működésében. Ezeket más könyvtárakkal, vagy saját megvalósítással kell helyettesíteni. Ha teljesen saját implementációval helyettesíti a {\fontfamily{cmtt}\selectfont vfc} keretrendszert, a következő osztályoknak, metódusoknak kell szerepelniük a megfelelő API függvényekkel:
\begin{itemize}
    \item Elemi típusok
    \begin{compactitem}
        \item 32 bites egész szám
        \item 32 bites lebegőpontos törtszám
        \item méter
        \item radián
    \end{compactitem}

    \item Matematikai függvények
    \begin{compactitem}
        \item $\pi$ értéke 32 bites lebegőpontos törtszámként
        \item {\fontfamily{cmtt}\selectfont sin(float32\_t f\_value)} - szinusz függvény
        \item {\fontfamily{cmtt}\selectfont cos(float32\_t f\_value)} - koszinusz függvény
        \item {\fontfamily{cmtt}\selectfont abs(float32\_t f\_value)} - abszolútérték függvény
        \item {\fontfamily{cmtt}\selectfont min(float32\_t f\_a, float32\_t f\_b)} - minimum függvény
        \item {\fontfamily{cmtt}\selectfont max(float32\_t f\_a, float32\_t f\_b)} - maximum függvény
    \end{compactitem}

    \item Tárolók
    \begin{compactitem}
        \item {\fontfamily{cmtt}\selectfont TFixedVector<ValueType, MaxLength>} - fix hosszúságú vektor
        \item {\fontfamily{cmtt}\selectfont TFixedMap<KeyType, ValueType, MaxLength>} - fix hosszúságú \emph{map}
    \end{compactitem}
\end{itemize}

\begin{note}
    Minden felhasználó által definiált elemi típushoz szükséges a négy alapművelet ($+, -, \times, \div$) és a logikai operátorok ($=, \neq, <, >, \leq, \geq$) definiálása is.
\end{note}

\subsection{Hibaüzenetek}

A könyvtár osztályai nem sok helyen váltanak ki hibaüzenetet, csak ahol mindenképp szükséges. Az üzenetek és az őket kiváltó függvényei a következők:
\begin{itemize}
    \item {\fontfamily{cmtt}\selectfont CMinBinaryHeap::removeMinimum} - "removeMinimum was called on empty heap."
    \item {\fontfamily{cmtt}\selectfont CMinBinaryHeap::getMinimum} - "getMinimum was called on empty heap."
    \item {\fontfamily{cmtt}\selectfont CPathDatabase::getDirectPathEndingWith} - "Requested path not found!"
    \item {\fontfamily{cmtt}\selectfont CPathDatabase::getConnectingMovesEndingWith} - "Requested moves not found!"
\end{itemize}

Ezen hibák esetén a program csak a tesztkörnyezetben fejezi be a működését, a valódi beágyazott környezetben nem, így máshogy is gondoskodni kell a hibakezelésről. Lehetőség ilyenkor helyettesítő adatot (például {\fontfamily{cmtt}\selectfont nullptr} \emph{pointer-ek} esetében) adni visszatérési értéknek. Ettől sem feltétlenül egyszerűbb a feladat, hiszen a továbbadott értéket is használhatja egy másik objektum. Erre lehet példa a következő kódrészlet:
\\

\lstset{caption={Többlépcsős hibakezelés}, label=src:altError}
\begin{lstlisting}[language={C++}]
VFC_ASSERT2(false == m_openList.isEmpty(), "Getter function called on an empty heap.");

if (m_openList.isEmpty())
{
    return nullptr;
}

return m_openList.getMinimum().getPrimitiveLeadingToNode();
\end{lstlisting}

Ebben a példában a függvény eredményét is vizsgálni kell majd, mert a {\fontfamily{cmtt}\selectfont nullptr}-nek nem lesznek használható értékei.

\section{A vizualizációs szoftver}
Az útvonaltervezőt, és ezzel együtt a \emph{Waypoint Module-t} is vizualizációs szoftver segítségével teszteltem. Így lehetőség van tesztelni a legtöbb funkciót anélkül, hogy egy valós járműbe integrálnánk a projektet. Segítség még, hogy nem kell felhasználni más \emph{subsystem-eket}, hogy működni tudjon a tervezés, így önállóan csak az útvonaltervezőre lehet fókuszálni. A teljes vizualizációs szoftver C++ nyelven a Qt keretrendszerrel készült.

Ebben a részben bemutatom, hogyan lehet használni a szoftvert, a felhasználói felület elemei milyen funkcióval bírnak, és milyen beállítások lehetségesek a járművekhez és a tervezéshez. Kitérek még a \emph{scene-ek} készítésére, mentésére és betöltésére is.

\subsection{A felhasználó felület elemei}
Amikor belép a felhasználó az alkalmazásba, minden alapállapotban van. Ez azt jelenti, hogy:
\begin{itemize}
    \item A startpont a $(0,0,0)$, a célpont pedig az $(1, 0, 0)$ pont
    \item A jármű kontúrja a {\fontfamily{cmtt}\selectfont black\_lion.json}, nincs biztonsági távolság beállítva
    \item Csak a start- és a célpontban lévő jármű van vizualizálva
    \item Nincsenek akadályok a \emph{scene-en}
\end{itemize}

A felhasználói felület 4 részre osztható fel. Ezek pedig a következők:
\begin{itemize}
    \item Eszköztár, a képernyő felső részén
    \item Paraméterbeállítások, a képernyő bal oldalán
    \item Vizualizációs \emph{canvas}, középen
    \item Eredmények, a képernyő jobb oldalán
\end{itemize}

\begin{figure}[H]
    \centering
	\includegraphics[width=1\textwidth]{qt_start}
	\caption{A felhasználói felület belépéskor}
	\label{fig:qtOnEnter}
\end{figure}

Az eszköztárban találhatók az alapvető funkciókat végző gombok. Ezek segítségével lehet útvonaltervezést indítani (\emph{"Plan"}), törölni az objektumokat a \emph{scene-ről} (\emph{"Clear"}), menteni illetve betölteni (\emph{"Load", "Save"}) és exportálni a tervezett útvonalat (\emph{"Export path"}). A program megadott formátumú {\fontfamily{cmtt}\selectfont .json} kiterjesztésű \emph{file-okat} képes kezelni, a mentés is ilyen formában történik. Az exportálás két állományba történik: {\fontfamily{cmtt}\selectfont planned\_path.csv} és {\fontfamily{cmtt}\selectfont smoothed\_path.csv}. Az elsőben az eredeti utat menti a szoftver, míg a másodikba a simított útvonalat, amelyet a járművel zökkenőmentesebben lehet megtenni. Az eredeti út a \emph{file-ban} primitívek formájában, míg a simított útvonal diszkrét pontokként szerepel.

Az eredményeket ábrázoló grafikonok több oldalon vannak ábrázolva. Az elsőn a különböző tervezési költségek szerepelnek (mind az idő függvényében):
\begin{itemize}
    \item Visszamaradó ívhossz
    \item Visszamaradó \emph{running cost}
    \item A heurisztika értéke
\end{itemize}
Ezek alapján lehet vizsgálni, hogy minden \emph{Motion Primitive} után mennyit közelített a célponthoz a jármű. A második és harmadik grafikon a simított útvonalakon vizsgálja a jármű és a kormányzott kerekei szögének változását ugyanúgy az idő függvényében.

A paraméterbeállításokról és a vizualizációs \emph{canvas-ról} a következő alpontban beszélek részletesebben.

\subsection{Útvonaltervezés beállítása, indítása}
Az útvonaltervezéshez szükséges beállítások mind a képernyő bal oldalán, a paraméterbeállításoknál találhatók. Az első modulban az általános beállítások szerepelnek. Ezek  három oldalra vannak osztva:
\begin{itemize}
    \item \emph{"Vehicle"} - járműbeállítások
    \item \emph{"Visualization"} - vizualizációs beállítások
    \item \emph{"Planner"} - az útvonaltervező beállításai
\end{itemize}
A járműbeállításokban két lehetőség van a személyre szabáshoz: a jármű típusának kiválasztása és a kontúrtípus beállítása. A kontúrtípus lehet a jármű körvonalával megegyező (\emph{Original}) vagy biztonsági távolsággal ellátott (\emph{Safety}).
\begin{note}
    Ezeket a távolságértékeket a \emph{"Parameters"} menü \emph{"Misc"} oldalán tudjuk állítani. Első, hátsó, jobb- és baloldali értékek megadására van lehetőség (méterben).
\end{note}
\clearpage

Háromféle járműtípus kiválasztására van lehetőség:
\begin{itemize}
    \item Black Lion (Peugeot 508)
    \item Rhino (Range Rover)
    \item Grand Wagoneer (Jeep Grand Wagoneer)
\end{itemize}
Ezeket a kontúrokat is {\fontfamily{cmtt}\selectfont .json} \emph{file-okban} tárolja a program, melyekben a jármű körvonalának bizonyos $x$ és $y$ koordinátái szerepelnek a $(0, 0)$ ponthoz képest.
\begin{figure}[H]
    \centering
	\includegraphics[width=300px]{vehicle_contours}
	\caption{A választható járművek kontúrjai}
	\label{fig:vehicleContours}
\end{figure}

A vizualizációs beállításokkal lehet vezérelni, hogy mit és mit ne jelenítsen meg a program a \emph{canvas-on}. Minden opcióhoz egy \emph{checkbox} tartozik, ezek a következők:
\begin{itemize}
    \item start- és célponton lévő jármű kontúrja
    \item \emph{waypoint-ok} kontúrjai
    \item az egér aktuális pozíciója
    \item a járművek visszapillantó-tükrei
    \item a megtett út kontúrja (primitívenként)
    \item a tervezett út cseréje simított útra
\end{itemize}
A startpozíció rózsaszínnel, a célpozíció világoskékkel és a felhasznált \emph{waypoint-ok} zölddel vannak ábrázolva. Ha be van kapcsolva a megtett út kontúrjának vizualizálása, akkor minden primitív végénél megjelenik egy másolata a jármű körvonalának. Ennek kitöltése sötétszürke ha a mozdulat előre történik, különben pedig világosszürke. A képernyő alján vannak további vizualizációs lehetőségek is, kirajzolhatók a \emph{node expansion-ök} és mozdulatonként is ábrázolható az útvonal. Ezeket a megfelelő gombokkal kell bekapcsolni. Lenyomásra kirajzolja őket a program, ha rendelkezésre állnak, és a \emph{"Reset"} gombbal lehet eltüntetni.

Az útvonaltervezés beállításainak utolsó oldalán szerepelnek a \emph{planner} beállítási lehetőségei. Itt van lehetőség be- és kikapcsolni bizonyos funkciókat, mint:
\begin{itemize}
    \item Fordított tervezés
    \item Előző tervezések vizualizálása
    \item \emph{Waypoint Module}
    \item \emph{Post-Smoothing} algoritmusok váltása
\end{itemize}

A fordított tervezésnél a cél- és a startpontok megcserélődnek, és visszafelé történik a tervezés. Hasznos lehet több esetben, mert a tervező algoritmus kevesebb iterációból talál ki két objektum közül (például 90°-os parkolásnál), mint amennyiből betalál a köztük lévő helyre. A \emph{Waypoint Module} aktiválásával kapcsolhatja be a felhasználó a többrészes tervezést. Ha be van kapcsolva az előző tervezések vizualizálása, lehetőség van megadni, hogy mennyi előző utat rajzoljon ki a szoftver (1-10 között). Ezek az utak a piros helyett szürkével jelennek meg.
\begin{note}
    A fordított tervezés és a \emph{Waypoint Module} együtt nem használható.
\end{note}

A paramétereknél és a bemeneti adatoknál lehet további beállításokat megadni. Itt állítható a heurisztika típusa, a különböző típusspecifikus paramétereik és az irányváltási büntetőérték is. A bemeneti adatoknál adható meg a start- és a célpont, és a hozzájuk tartozó tolerancia. Ha az algoritmus ezen belüli pontot talál, a tervezés sikeresnek tekinthető. Itt állíthatók a \emph{Waypoint Module-höz} tartozó paraméterek is. Az ezek a beállítások a \emph{"Parameters"} rész \emph{"Waypoint Module"} oldalán találhatók. Ha minden elem beállításra került, a \emph{"Plan"} gomb megnyomásával indítható a tervező.
\begin{figure}[H]
    \centering
	\includegraphics[width=1\textwidth]{qt_planning}
	\caption{Egy tervezés eredménye}
	\label{fig:planningResult}
\end{figure}

\subsection{\emph{Scene-ek} készítése, betöltése, mentése}
A következő része a felhasználói felületnek a vizualizációs \emph{canvas} és a hozzá tartozó objektumlista. Ezen a modulon jelennek meg a kirajzolt tárgyak, járművek és útvonalak. Ezek közül az egyik legfontosabb funkció az objektumok megjelenítése, amiket lehetőség van \emph{file-ból} betölteni vagy kézzel rajzolni.

A kézzel való objektumkészítés megkezdéséhez az 'O' billentyűt kell megnyomni. Ilyenkor a program átvált \emph{Object Placement} módba. Ebben a módban minden eddig tárgyalt funkció működik, és kiegészül még az objektumkészítéssel is. A rajzoláshoz a bal vagy középső egérgombbal kell a \emph{canvas-ra} kattintani, ilyenkor lekerül egy pontja az objektumnak. Maximum 16 pontból állhat egy ilyen rajzolt objektum, ezeket a pontokat a program lehelyezési sorrendben automatikusan összeköti. Ha kevesebb pontból álló akadályt szeretne készíteni a felhasználó, az \emph{'ENTER'} billentyű megnyomásával elkészül az objektum.

Lehetőség van magas, illetve alacsony objektumok készítésére is. Az alacsony akadályokon a jármű képes áthaladni, míg a magasakon nem. Lehetséges kombinált objektum létrehozására is, ilyenkor a következő szabályok szerint működik a magasság eldöntése:
\begin{itemize}
    \item Ha két szomszédos pont mindegyike alacsony, a köztük lévő vonal is alacsony
    \item Különben az őket összekötő vonal magas
\end{itemize}

A rajzoláskor alacsony pontot a középső egérgombbal lehet lehelyezni, míg magasat a ballal. A \emph{canvas-on} az alacsony vonalak kékkel, a magasak feketével és az éppen rajzoltak pedig világoszölddel jelennek meg. Egy elkészült objektum látható az \emph{"Obstacles on scene"} oldalon is, a felhasználói felület jobb oldalán. Itt kijelölhető akármelyik lehelyezett akadály, ilyenkor a \emph{canvas-on} piros színre vált. A kijelölt objektumot a \emph{'DEL'} billentyűvel lehet törölni. A felhasználó az \emph{'ESC'} billentyűvel tud visszatérni az alap módba. 
\begin{figure}[H]
    \centering
	\includegraphics[width=1\textwidth]{qt_objects}
	\caption{Alacsony, magas, kombinált és aktív objektumok}
	\label{fig:objectTypes}
\end{figure}
Előre elkészített kör és téglalap lehelyezésére is van lehetőség. Ezt a \emph{"Create obstacle"} oldalon lehet megtenni, miután megadta a felhasználó a megfelelő adatokat. Kör esetén ez a középpont $x$ és $y$ koordinátája, illetve a sugara. Téglalapnál pedig a középpont $x$ és $y$ koordinátája, a szélessége és magassága, illetve az elforgatása.

A kész objektumok mozgatására is van lehetőség, \emph{Object Movement} módba az 'M' billentyű lenyomásával lehet váltani. Ilyenkor az akadályokat \emph{drag-and-drop} módszerrel lehet mozgatni a teljes \emph{canvas-on}. A legutolsó elmozdított objektum automatikusan kijelölésre kerül a jobboldali listában is.

A másik módja a \emph{scene-ek} használatának a \emph{file-ból} való betöltés. Ehhez egy megfelelő formátumú {\fontfamily{cmtt}\selectfont .json} állományt kell elkészíteni, mely a következőképpen épül fel:
\clearpage

\lstset{caption={Bemeneti {\fontfamily{cmtt}\selectfont .json} állomány felépítése}, label=src:inputJson}
\begin{lstlisting}[language={json}]
{
    "PmpParameters": "default",
    "CommonParameters": "default",
    "Frames": {
        "0": {
            "PlanningRequest": {
                "m_startDirection": "startBackward",
                "m_startPosture": {
                    "m_pose": [0.0, 0.0, 0.0],
                    "m_curvature": 0.0
                },
                "m_targetArea": {
                    "m_targetPosture": {
                        "m_pose": [1.0, 0.0, 0.0],
                        "m_curvature": 0.0
                    },
                    "m_lateralTolerance": 0.03,
                    "m_longitudinalTolerance": 0.03,
                    "m_orientationTolerance": 0.049742,
                    "m_targetDirection": "startBackward"
                }
            },
            "NfmAggregatedPolygonObjects": [
                {
                    "nfmPolygonObjectNodes": [
                        {
                            "m_x": 1.869132,
                            "m_y": -3.324296,
                            "m_height": "HIGH"
                        },
                        {
                            "m_x": 2.747248,
                            "m_y": -2.333279,
                            "m_height": "HIGH"
                        }
                    ]
                }
            ]}}}
\end{lstlisting}

Ezeket a \emph{file-okat} nem ajánlott kézzel elkészíteni, hiszen egy már megrajzolt \emph{scene} mentésére is van lehetőség. A \emph{"Save"} gomb lenyomása után ki kell választani a megfelelő mappát, ahova a \emph{file-t} szeretné menteni a felhasználó, majd a \emph{"Save"} lehetőségre kattintani a felugró ablakban. A \emph{file} névhez nem szükséges hozzáírni a {\fontfamily{cmtt}\selectfont .json} kiterjesztést, a program automatikusan kiegészíti azt. Ha viszont hozzáírja a felhasználó, akkor nem kerül rá kétszer.

Minden \emph{input-hoz} meg van adva egy intervallum, amelyben a hozzájuk kötött értékek mozoghatnak. Ezeket a következő táblázatban ismertetem.

\begin{table}[htb]
	\centering
	\begin{tabular}{ | c | c | c | }
		\hline
		\multirow{2}{*}{\textbf{Paraméter}} & \multicolumn{2}{ c | }{\textbf{Limit}} \\
		\cline{2-3}
		& Min. & Max\\
		\hline \hline		
		Téglalap magassága / szélessége & 0 & 1000 \\
		\hline
		Kör sugara & 0 & 1000  \\
		\hline
		Kör és téglalap középpontja & $(-1000,-1000)$ & $(1000, 1000)$ \\
		\hline 
		Forgatások (téglalap, \emph{start}, \emph{target}, $\sigma\theta$) & 0 & 360 \\
		\hline 
		\emph{Start-, target} pozíciók & $(-10,-10)$ & $(10,10)$ \\
		\hline 
		Minimum mozgás egyenes irányba & 0 & 10 \\
		\hline 
	    $\sigma$ értékek (kivéve $\theta$) & 0 & 10 \\
		\hline 
	\end{tabular}
	\caption{Legfontosabb paraméterek limitációi}
	\label{tab:paramLimits}
\end{table}