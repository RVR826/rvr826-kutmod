\documentclass[journal]{IEEEtran}

% *** CITATION PACKAGES ***
%
\usepackage[
backend=biber,
style=numeric
]{biblatex}
\addbibresource{main.bib}
% cite.sty was written by Donald Arseneau
% V1.6 and later of IEEEtran pre-defines the format of the cite.sty package
% \cite{} output to follow that of the IEEE. Loading the cite package will
% result in citation numbers being automatically sorted and properly
% "compressed/ranged". e.g., [1], [9], [2], [7], [5], [6] without using
% cite.sty will become [1], [2], [5]--[7], [9] using cite.sty. cite.sty's
% \cite will automatically add leading space, if needed. Use cite.sty's
% noadjust option (cite.sty V3.8 and later) if you want to turn this off
% such as if a citation ever needs to be enclosed in parenthesis.
% cite.sty is already installed on most LaTeX systems. Be sure and use
% version 5.0 (2009-03-20) and later if using hyperref.sty.
% The latest version can be obtained at:
% http://www.ctan.org/pkg/cite
% The documentation is contained in the cite.sty file itself.

% *** GRAPHICS RELATED PACKAGES ***
%
\ifCLASSINFOpdf
  \usepackage[pdftex]{graphicx}
  % declare the path(s) where your graphic files are
  \graphicspath{{../pdf/}{../jpeg/}}
  % and their extensions so you won't have to specify these with
  % every instance of \includegraphics
  \DeclareGraphicsExtensions{.pdf,.jpeg,.png}
\else
  % or other class option (dvipsone, dvipdf, if not using dvips). graphicx
  % will default to the driver specified in the system graphics.cfg if no
  % driver is specified.
  % \usepackage[dvips]{graphicx}
  % declare the path(s) where your graphic files are
  % \graphicspath{{../eps/}}
  % and their extensions so you won't have to specify these with
  % every instance of \includegraphics
  % \DeclareGraphicsExtensions{.eps}
\fi
% graphicx was written by David Carlisle and Sebastian Rahtz. It is
% required if you want graphics, photos, etc. graphicx.sty is already
% installed on most LaTeX systems. The latest version and documentation
% can be obtained at: 
% http://www.ctan.org/pkg/graphicx
% Another good source of documentation is "Using Imported Graphics in
% LaTeX2e" by Keith Reckdahl which can be found at:
% http://www.ctan.org/pkg/epslatex
%
% latex, and pdflatex in dvi mode, support graphics in encapsulated
% postscript (.eps) format. pdflatex in pdf mode supports graphics
% in .pdf, .jpeg, .png and .mps (metapost) formats. Users should ensure
% that all non-photo figures use a vector format (.eps, .pdf, .mps) and
% not a bitmapped formats (.jpeg, .png). The IEEE frowns on bitmapped formats
% which can result in "jaggedy"/blurry rendering of lines and letters as
% well as large increases in file sizes.
%
% You can find documentation about the pdfTeX application at:
% http://www.tug.org/applications/pdftex





% *** MATH PACKAGES ***
%
\usepackage{amsmath}
\usepackage{amsfonts}
% A popular package from the American Mathematical Society that provides
% many useful and powerful commands for dealing with mathematics.
%
% Note that the amsmath package sets \interdisplaylinepenalty to 10000
% thus preventing page breaks from occurring within multiline equations. Use:
%\interdisplaylinepenalty=2500
% after loading amsmath to restore such page breaks as IEEEtran.cls normally
% does. amsmath.sty is already installed on most LaTeX systems. The latest
% version and documentation can be obtained at:
% http://www.ctan.org/pkg/amsmath





% *** SPECIALIZED LIST PACKAGES ***
%
%\usepackage{algorithmic}
% algorithmic.sty was written by Peter Williams and Rogerio Brito.
% This package provides an algorithmic environment fo describing algorithms.
% You can use the algorithmic environment in-text or within a figure
% environment to provide for a floating algorithm. Do NOT use the algorithm
% floating environment provided by algorithm.sty (by the same authors) or
% algorithm2e.sty (by Christophe Fiorio) as the IEEE does not use dedicated
% algorithm float types and packages that provide these will not provide
% correct IEEE style captions. The latest version and documentation of
% algorithmic.sty can be obtained at:
% http://www.ctan.org/pkg/algorithms
% Also of interest may be the (relatively newer and more customizable)
% algorithmicx.sty package by Szasz Janos:
% http://www.ctan.org/pkg/algorithmicx




% *** ALIGNMENT PACKAGES ***
%
%\usepackage{array}
% Frank Mittelbach's and David Carlisle's array.sty patches and improves
% the standard LaTeX2e array and tabular environments to provide better
% appearance and additional user controls. As the default LaTeX2e table
% generation code is lacking to the point of almost being broken with
% respect to the quality of the end results, all users are strongly
% advised to use an enhanced (at the very least that provided by array.sty)
% set of table tools. array.sty is already installed on most systems. The
% latest version and documentation can be obtained at:
% http://www.ctan.org/pkg/array


% IEEEtran contains the IEEEeqnarray family of commands that can be used to
% generate multiline equations as well as matrices, tables, etc., of high
% quality.




% *** SUBFIGURE PACKAGES ***
%\ifCLASSOPTIONcompsoc
%  \usepackage[caption=false,font=normalsize,labelfont=sf,textfont=sf]{subfig}
%\else
%  \usepackage[caption=false,font=footnotesize]{subfig}
%\fi
% subfig.sty, written by Steven Douglas Cochran, is the modern replacement
% for subfigure.sty, the latter of which is no longer maintained and is
% incompatible with some LaTeX packages including fixltx2e. However,
% subfig.sty requires and automatically loads Axel Sommerfeldt's caption.sty
% which will override IEEEtran.cls' handling of captions and this will result
% in non-IEEE style figure/table captions. To prevent this problem, be sure
% and invoke subfig.sty's "caption=false" package option (available since
% subfig.sty version 1.3, 2005/06/28) as this is will preserve IEEEtran.cls
% handling of captions.
% Note that the Computer Society format requires a larger sans serif font
% than the serif footnote size font used in traditional IEEE formatting
% and thus the need to invoke different subfig.sty package options depending
% on whether compsoc mode has been enabled.
%
% The latest version and documentation of subfig.sty can be obtained at:
% http://www.ctan.org/pkg/subfig




% *** FLOAT PACKAGES ***
%
%\usepackage{fixltx2e}
% fixltx2e, the successor to the earlier fix2col.sty, was written by
% Frank Mittelbach and David Carlisle. This package corrects a few problems
% in the LaTeX2e kernel, the most notable of which is that in current
% LaTeX2e releases, the ordering of single and double column floats is not
% guaranteed to be preserved. Thus, an unpatched LaTeX2e can allow a
% single column figure to be placed prior to an earlier double column
% figure.
% Be aware that LaTeX2e kernels dated 2015 and later have fixltx2e.sty's
% corrections already built into the system in which case a warning will
% be issued if an attempt is made to load fixltx2e.sty as it is no longer
% needed.
% The latest version and documentation can be found at:
% http://www.ctan.org/pkg/fixltx2e


%\usepackage{stfloats}
% stfloats.sty was written by Sigitas Tolusis. This package gives LaTeX2e
% the ability to do double column floats at the bottom of the page as well
% as the top. (e.g., "\begin{figure*}[!b]" is not normally possible in
% LaTeX2e). It also provides a command:
%\fnbelowfloat
% to enable the placement of footnotes below bottom floats (the standard
% LaTeX2e kernel puts them above bottom floats). This is an invasive package
% which rewrites many portions of the LaTeX2e float routines. It may not work
% with other packages that modify the LaTeX2e float routines. The latest
% version and documentation can be obtained at:
% http://www.ctan.org/pkg/stfloats
% Do not use the stfloats baselinefloat ability as the IEEE does not allow
% \baselineskip to stretch. Authors submitting work to the IEEE should note
% that the IEEE rarely uses double column equations and that authors should try
% to avoid such use. Do not be tempted to use the cuted.sty or midfloat.sty
% packages (also by Sigitas Tolusis) as the IEEE does not format its papers in
% such ways.
% Do not attempt to use stfloats with fixltx2e as they are incompatible.
% Instead, use Morten Hogholm'a dblfloatfix which combines the features
% of both fixltx2e and stfloats:
%
% \usepackage{dblfloatfix}
% The latest version can be found at:
% http://www.ctan.org/pkg/dblfloatfix




%\ifCLASSOPTIONcaptionsoff
%  \usepackage[nomarkers]{endfloat}
% \let\MYoriglatexcaption\caption
% \renewcommand{\caption}[2][\relax]{\MYoriglatexcaption[#2]{#2}}
%\fi
% endfloat.sty was written by James Darrell McCauley, Jeff Goldberg and 
% Axel Sommerfeldt. This package may be useful when used in conjunction with 
% IEEEtran.cls'  captionsoff option. Some IEEE journals/societies require that
% submissions have lists of figures/tables at the end of the paper and that
% figures/tables without any captions are placed on a page by themselves at
% the end of the document. If needed, the draftcls IEEEtran class option or
% \CLASSINPUTbaselinestretch interface can be used to increase the line
% spacing as well. Be sure and use the nomarkers option of endfloat to
% prevent endfloat from "marking" where the figures would have been placed
% in the text. The two hack lines of code above are a slight modification of
% that suggested by in the endfloat docs (section 8.4.1) to ensure that
% the full captions always appear in the list of figures/tables - even if
% the user used the short optional argument of \caption[]{}.
% IEEE papers do not typically make use of \caption[]'s optional argument,
% so this should not be an issue. A similar trick can be used to disable
% captions of packages such as subfig.sty that lack options to turn off
% the subcaptions:
% For subfig.sty:
% \let\MYorigsubfloat\subfloat
% \renewcommand{\subfloat}[2][\relax]{\MYorigsubfloat[]{#2}}
% However, the above trick will not work if both optional arguments of
% the \subfloat command are used. Furthermore, there needs to be a
% description of each subfigure *somewhere* and endfloat does not add
% subfigure captions to its list of figures. Thus, the best approach is to
% avoid the use of subfigure captions (many IEEE journals avoid them anyway)
% and instead reference/explain all the subfigures within the main caption.
% The latest version of endfloat.sty and its documentation can obtained at:
% http://www.ctan.org/pkg/endfloat
%
% The IEEEtran \ifCLASSOPTIONcaptionsoff conditional can also be used
% later in the document, say, to conditionally put the References on a 
% page by themselves.




% *** PDF, URL AND HYPERLINK PACKAGES ***
%
%\usepackage{url}
% url.sty was written by Donald Arseneau. It provides better support for
% handling and breaking URLs. url.sty is already installed on most LaTeX
% systems. The latest version and documentation can be obtained at:
% http://www.ctan.org/pkg/url
% Basically, \url{my_url_here}.




% *** Do not adjust lengths that control margins, column widths, etc. ***
% *** Do not use packages that alter fonts (such as pslatex).         ***
% There should be no need to do such things with IEEEtran.cls V1.6 and later.
% (Unless specifically asked to do so by the journal or conference you plan
% to submit to, of course. )


% correct bad hyphenation here
\hyphenation{op-tical net-works semi-conduc-tor}


\begin{document}
%
% paper title
% Titles are generally capitalized except for words such as a, an, and, as,
% at, but, by, for, in, nor, of, on, or, the, to and up, which are usually
% not capitalized unless they are the first or last word of the title.
% Linebreaks \\ can be used within to get better formatting as desired.
% Do not put math or special symbols in the title.
\title{Benchmark of an LU decomposition algorithm in Python}
%
%
% author names and IEEE memberships
% note positions of commas and nonbreaking spaces ( ~ ) LaTeX will not break
% a structure at a ~ so this keeps an author's name from being broken across
% two lines.
% use \thanks{} to gain access to the first footnote area
% a separate \thanks must be used for each paragraph as LaTeX2e's \thanks
% was not built to handle multiple paragraphs
%

\author{Márton Papp,~\IEEEmembership{Student,~ELTE FI,}
        Ákos Szabó,~\IEEEmembership{Student,~ELTE FI,}}

% note the % following the last \IEEEmembership and also \thanks - 
% these prevent an unwanted space from occurring between the last author name
% and the end of the author line. i.e., if you had this:
% 
% \author{....lastname \thanks{...} \thanks{...} }
%                     ^------------^------------^----Do not want these spaces!
%
% a space would be appended to the last name and could cause every name on that
% line to be shifted left slightly. This is one of those "LaTeX things". For
% instance, "\textbf{A} \textbf{B}" will typeset as "A B" not "AB". To get
% "AB" then you have to do: "\textbf{A}\textbf{B}"
% \thanks is no different in this regard, so shield the last } of each \thanks
% that ends a line with a % and do not let a space in before the next \thanks.
% Spaces after \IEEEmembership other than the last one are OK (and needed) as
% you are supposed to have spaces between the names. For what it is worth,
% this is a minor point as most people would not even notice if the said evil
% space somehow managed to creep in.



% The paper headers
\markboth{Journal of Research Methodology Task 2\ Class Files,~Vol.~1, No.~1, November~2025}%
{Shell \MakeLowercase{\textit{et al.}}: Bare Demo of IEEEtran.cls for IEEE Journals}
% The only time the second header will appear is for the odd numbered pages
% after the title page when using the twoside option.
% 
% *** Note that you probably will NOT want to include the author's ***
% *** name in the headers of peer review papers.                   ***
% You can use \ifCLASSOPTIONpeerreview for conditional compilation here if
% you desire.




% If you want to put a publisher's ID mark on the page you can do it like
% this:
%\IEEEpubid{0000--0000/00\$00.00~\copyright~2015 IEEE}
% Remember, if you use this you must call \IEEEpubidadjcol in the second
% column for its text to clear the IEEEpubid mark.



% use for special paper notices
%\IEEEspecialpapernotice{(Invited Paper)}




% make the title area
\maketitle

% As a general rule, do not put math, special symbols or citations
% in the abstract or keywords.
\begin{abstract}
This study investigates the computational performance and numerical stability of the LU Decomposition algorithm, implemented using the Doolittle method with partial pivoting in Python.
Benchmarking was performed on randomly generated square matrices of increasing size, and results showed the expected cubic time complexity of $O(n^3)$, with runtime growing proportionally to matrix size.
Despite being implemented without low-level optimizations, the custom algorithm consistently produced decompositions satisfying $PA\approx LU$ with negligible residual errors, demonstrating high numerical reliability.
The findings confirm that even a pure NumPy-based implementation can achieve accurate results, though at a significant performance cost relative to optimized linear algebra libraries or parallelised algorithms \cite{KAYA2005179}.
\end{abstract}

% Note that keywords are not normally used for peerreview papers.
\begin{IEEEkeywords}
LU Decomposition, Numerical Algorithms, Marices, Python, Numpy
\end{IEEEkeywords}






% For peer review papers, you can put extra information on the cover
% page as needed:
% \ifCLASSOPTIONpeerreview
% \begin{center} \bfseries EDICS Category: 3-BBND \end{center}
% \fi
%
% For peerreview papers, this IEEEtran command inserts a page break and
% creates the second title. It will be ignored for other modes.
\IEEEpeerreviewmaketitle

\section{Introduction}

Numerical linear algebra forms a fundamental component of modern scientific computing, providing the mathematical and algorithmic tools needed to solve systems of equations, perform optimization, model physical phenomena, and process large volumes of observational data. Among its core operations, the factorization of matrices plays a crucial role in transforming complex algebraic problems into forms that are easier to solve or analyse. One of the most widely used and well-established matrix factorizations is the $LU$ decomposition, in which a given square matrix 
$A$ is expressed as the product of a lower-triangular matrix $L$ and an upper-triangular matrix $U$, optionally accompanied by a permutation matrix $P$ to ensure numerical stability through row pivoting.
\begin{equation}
    PA = LU
\end{equation}

The $LU$ decomposition underlies many essential numerical techniques. It provides an efficient method for solving linear systems of the form $Ax=b$, which appear ubiquitously in computational physics, engineering simulations, and data reduction pipelines. 
Once the matrix $A$ has been factored into $LU$, multiple right-hand sides can be solved at relatively low additional cost using forward and backward substitution. In contrast to direct matrix inversion, LU-based solvers tend to be both faster and numerically more stable. This makes $LU$ decomposition a preferred approach in large-scale simulations, iterative refinement procedures, and as a building block in more sophisticated algorithms such as matrix inversion, determinant evaluation, and preconditioners for iterative solvers.

In computer science and applied computing, $LU$ decomposition plays a pivotal role as a fundamental tool for solving linear systems, optimizing algorithms, and supporting a wide range of computational methods.
For example, $LU$ factorisation is routinely used in constraint solving, network flow computations \cite{bandiyah2025implementing}, numerical optimization routines such as interior-point methods, and in the preprocessing of matrices for direct or iterative solvers.
Furthermore, many modern algorithms such as Kalman filters \cite{simon2001kalman}, Gaussian elimination for dense systems, and various least-squares formulations, either rely directly on $LU$ decomposition or benefit from its ability to provide fast, reusable factorizations for multiple right-hand sides.

Despite its conceptual simplicity, the computational cost of $LU$ decomposition scales as $O(n^3)$, making performance evaluation and algorithmic optimisation particularly relevant for large problem sizes. Furthermore, numerical stability and floating-point round-off behaviour can vary depending on the implementation strategy and pivoting scheme. For these reasons, analysing and benchmarking a custom $LU$ implementation provides valuable insights into both algorithmic characteristics and the practical limitations of high-level numerical code, especially when written in languages such as Python where low-level optimisations are not automatically guaranteed.

\section{Methodology}

This study evaluates the computational performance and numerical accuracy of a custom LU decomposition implementation written in Python using NumPy.
The methodological design consists of two main components: 
\begin{enumerate}
  \item the specification of the hardware and software environment in which all experiments were executed
  \item the configuration of the benchmarking procedures, including matrix generation, timing strategy, and error measurement
\end{enumerate}
The goal is to ensure that results are both reproducible and representative of typical scientific computing workloads.

\subsection{Hardware and Software Environment}

All experiments were conducted on a single workstation to avoid variability introduced by differing system architectures. The machine used in this study featured the following hardware:
\begin{itemize}
  \item CPU: Intel Core i5-1135G6 2.4 GHz
  \item Cores: 4
  \item RAM: 8 GB (7.83 GB usable)
  \item OS: Windows 11 x64 Enterprise 25H2
\end{itemize}

All computations were performed in Python 3.14 using:
\begin{itemize}
  \item NumPy (version 2.3) for dense matrix operations
  \item The built-in performance counter for high-precision timing
  \item No additional numerical optimization libraries for the custom implementation
\end{itemize}

This setup intentionally reflects a typical high-level scientific computing environment where matrix operations are executed in Python without specialized low-level optimizations. The hardware characteristics are therefore directly reflected in the observed runtime behavior.

\subsection{Benchmarking Process}

The benchmarking framework is designed to measure execution time, runtime variability, and numerical stability of the LU decomposition across a range of matrix sizes. Two key design choices lead the structure of the benchmarks:
\begin{itemize}
  \item matrix size selection
  \item timing strategy
\end{itemize}

Square matrices of increasing dimensions were used to capture the algorithm’s $O(n^3)$ runtime scaling. Representative sizes were $n \in {10, 50, 250, 1000}$.
These values provide coverage across small, medium and large matrices with near-instant execution to cubic runtime dominant testcases.
Each matrix size was benchmarked over 100 independent trials. For each trial a fresh random matrix $A$ was generated, the timer was started. Then the algorithm was run, and the timer stopped immediately afterward.

To assess numerical stability, the Frobenius-norm residual was computed. This metric quantifies the deviation between the original matrix and its reconstructed form.
Residuals were collected across all trials and all matrix sizes, allowing comparison of floating-point accuracy as a function of problem scale.

\section{Results and Analysis}

This section presents the experimental results obtained from benchmarking the custom $LU$ decomposition algorithm implemented using NumPy. The analysis focuses on runtime characteristics and numerical stability across a range of matrix sizes.
The sample dataset was random square matrices generated by NumPy with the {\fontfamily{cmtt}\selectfont random.rand(n, n)} function for each $\mathbb{R}^{n \times n}$ input.

\subsection{Runtime Scaling with Matrix Size}

\begin{figure}
    \includegraphics[width=\linewidth]{rt_vs_msize.png}
    \caption{Mean runtime of LU decomposition for increasing matrix sizes.}
    \label{fig:runtime}
\end{figure}

Figure~\ref{fig:runtime} demonstrates a clear and rapid growth in computation time as the matrix dimension increases. 
The algorithm completes almost instantly for $n = 10$ and $n = 50$, but runtime rises sharply for larger matrices, 
reaching well over one second for $n = 1000$. This progression aligns with the theoretical cubic time complexity 
$O(n^3)$ of $LU$ decomposition: even moderate increases in matrix dimension lead to disproportionately large increases 
in runtime. The smooth monotonic growth also indicates that the implementation behaves consistently across the tested sizes,
without significant irregularities or performance anomalies.


\subsection{Distribution of Runtime Measurements}

\begin{figure}
    \centering
    \includegraphics[width=\linewidth]{distr_vs_msize.png}
    \caption{Boxplot of runtime variability over 100 runs per matrix size.}
    \label{fig:runtime_distribution}
\end{figure}

Figure~\ref{fig:runtime_distribution} displays the variability of runtimes across 100 repeated trials for each 
matrix size. For smaller matrices ($n = 10$ and $n = 50$), runtimes cluster tightly near zero with negligible 
dispersion, reflecting the low computational burden and minimal sensitivity to system noise. At $n = 250$, 
runtimes remain stable but show a visibly wider spread, consistent with increased computational workload. 
For $n = 1000$, the distribution widens substantially and includes several outliers exceeding 2.5 seconds. 
This indicates that large-scale decompositions are more susceptible to system-level fluctuations—such as cache behavior 
and OS scheduling, which become increasingly significant at higher computation loads.


\subsection{Residual Error Distribution per Matrix Size}

\begin{figure*}
    \centering
    \includegraphics[width=\linewidth]{res_error.png}
    \caption{Residual error histograms $\|PA - LU\|_F$ for matrix sizes $n = 10$, $50$, $250$, and $1000$.}
    \label{fig:residuals}
\end{figure*}

Figure~\ref{fig:residuals} shows separate histograms of the Frobenius-norm residual error $\|PA - LU\|_F$ for each tested 
matrix size. The results indicate that the custom $LU$ implementation maintains excellent numerical stability across all scales. 
Although the magnitude of the residual increases with matrix size, from approximately $10^{-16}$ at $n = 10$ to 
around $10^{-12}$ at $n = 1000$. These values remain extremely small relative to the size and magnitude of the matrices. 
This confirms that round-off error grows predictably with problem size but does not compromise accuracy. The distributions are 
tightly concentrated for each matrix dimension, demonstrating consistent behavior across trials and validating the robustness 
of the partial-pivoting approach for dense random matrices.

\section{Conclusion}
In this study, we implemented and benchmarked a custom $LU$ decomposition algorithm using Doolittle’s method with partial pivoting, focusing on both computational performance and numerical stability. The results demonstrate that the pure Python and NumPy-based implementation scales consistently with the theoretical $O(n^3)$
complexity, with runtime increasing sharply for larger matrices as expected. Despite the absence of low-level optimizations, the algorithm produced highly accurate decompositions across all tested matrix sizes, with residual errors remaining well within acceptable numerical bounds. 
These findings highlight both the strengths and limitations of high-level numerical programming: while suitable for small to medium-sized problems and educational purposes, performance constraints become prominent for large matrices, emphasizing the need for optimized libraries in production environments. 
Overall, the benchmarking confirms that the custom implementation behaves predictably and reliably, providing a useful reference point for understanding the computational characteristics of LU factorization.

% references section

% can use a bibliography generated by BibTeX as a .bbl file
% BibTeX documentation can be easily obtained at:
% http://mirror.ctan.org/biblio/bibtex/contrib/doc/
% The IEEEtran BibTeX style support page is at:
% http://www.michaelshell.org/tex/ieeetran/bibtex/
% argument is your BibTeX string definitions and bibliography database(s)
\printbibliography
%
% <OR> manually copy in the resultant .bbl file
% set second argument of \begin to the number of references
% (used to reserve space for the reference number labels box)
% \begin{thebibliography}{1}

% \bibitem{IEEEhowto:kopka}
% H.~Kopka and P.~W. Daly, \emph{A Guide to \LaTeX}, 3rd~ed.\hskip 1em plus
%   0.5em minus 0.4em\relax Harlow, England: Addison-Wesley, 1999.

% \end{thebibliography}

\end{document}