\documentclass[archE1,portrait]{baposter}

%841mm x 1189mm
\usepackage[font=small,labelfont=bf]{caption} % Required for specifying captions to tables and figures
\usepackage{amssymb}
\usepackage{booktabs} % Horizontal rules in tables
\usepackage{relsize} % Used for making text smaller in some places
\usepackage[urlcolor  = blue]{hyperref}
\graphicspath{{figures/}} % Directory in which figures are stored

\definecolor{bordercol}{RGB}{5,2,82} % Border color of content boxes
\definecolor{headercol1}{RGB}{5,2,82} % Background color for the header in the content boxes (left side)
\definecolor{headercol2}{RGB}{5,2,82} % Background color for the header in the content boxes (right side)
\definecolor{headerfontcol}{RGB}{255,255,255} % Text color for the header text in the content boxes
\definecolor{boxcolor}{RGB}{255,255,255} % Background color for the content in the content boxes

\begin{document}

\background{ % Set the background to an image (background.pdf)

}

\begin{poster}{
grid=false,
borderColor=bordercol, % Border color of content boxes
headerColorOne=headercol1, % Background color for the header in the content boxes (left side)
headerColorTwo=headercol1, % Background color for the header in the content boxes (right side)
headerFontColor=headerfontcol, % Text color for the header text in the content boxes
boxColorOne=boxcolor, % Background color for the content in the content boxes
headershape=roundedright, % Specify the rounded corner in the content box headers
headerfont=\Large\sf\bf, % Font modifiers for the text in the content box headers
textborder=rectangle,
background=user,
headerborder=open, % Change to closed for a line under the content box headers
boxshade=plain
}
{}
%
%----------------------------------------------------------------------------------------
%	TITLE AND AUTHOR NAME
%----------------------------------------------------------------------------------------
%
%\vspace{2em}
{
%\newline
Benchmark of an LU decomposition algorithm in Python} % Poster title
{\vspace{1em} Márton Papp, Ákos Szabó\\ % Author names
{\smaller ELTE FI}\\
\vspace{2em}
} % Author email addresses
%

%----------------------------------------------------------------------------------------
%	INTRODUCTION
%----------------------------------------------------------------------------------------

\headerbox{Introduction}{name=introduction,column=0,row=0, span=3}{
The LU decomposition ($PA=LU$) is a cornerstone of numerical linear algebra, essential for solving systems of linear equations ($Ax=b$) and inverting matrices.

\begin{itemize}
    \item \textbf{Importance:} It allows efficient solving for multiple right-hand sides compared to direct inversion. \cite{kaya2005}
    \item \textbf{Problem:} The algorithm has a cubic time complexity ($O(n^3)$), making performance critical for large datasets.
    \item \textbf{Goal:} This study benchmarks a custom "pure Python" implementation (Doolittle's method with partial pivoting) to evaluate runtime scaling and numerical stability without low-level optimizations.
\end{itemize}
}

%----------------------------------------------------------------------------------------
%	MATERIALS AND METHODS
%----------------------------------------------------------------------------------------

\headerbox{Materials and Methods}{name=methods,column=0,below=introduction, span=3}{

\textbf{Implementation Environment}
\begin{itemize}
    \item \textbf{Hardware:} Intel Core i5-1135G6 (2.4 GHz, 4 cores), 8GB RAM (Standard Workstation). [cite: 35-37]
    \item \textbf{Software:} Python 3.14 using NumPy 2.3 for dense matrix operations. [cite: 39, 42]
    \item \textbf{Algorithm:} Custom implementation of Doolittle's method with partial pivoting. No external optimization libraries were used. [cite: 6, 44]
\end{itemize}

\textbf{Benchmarking Protocol}
\begin{itemize}
    \item \textbf{Dataset:} Randomly generated square matrices ($A \in \mathbb{R}^{n \times n}$). [cite: 61]
    \item \textbf{Matrix Sizes:} $n \in \{10, 50, 250, 1000\}$ to capture execution scaling. 
    \item \textbf{Trials:} 100 independent runs per matrix size to measure variability. 
    \item \textbf{Error Metric:} Numerical stability assessed via Frobenius-norm residual: $||PA - LU||_F$ 
\end{itemize}
}

%----------------------------------------------------------------------------------------
%	RESULTS
%----------------------------------------------------------------------------------------
\headerbox{Conclusions}{name=conclusion,column=2,span=1,below=methods}{
The implementation confirms the expected $O(n^3)$ complexity while maintaining high numerical accuracy ($PA \approx LU$). Although suitable for educational benchmarking, production environments require optimized libraries for large-scale performance.
}

\headerbox{References}{name=references,column=2, below=conclusion}{

\smaller 
\renewcommand{\section}[2]{\vskip 0.05em}
\nocite{*} 

\bibliographystyle{unsrt}
\bibliography{references} 
\vspace{1em}
}

\headerbox{Results}{name=results, column=0 , span=2, below=methods, bottomaligned=references}{ 

\begin{center} 
    \begin{minipage}[t]{0.4\linewidth}
        \centering
        \includegraphics[width=0.95\linewidth]{figures/rt_vs_msize} 
        \captionof{figure}{Mean runtime vs. Matrix Size. The execution time follows the theoretical cubic growth ($O(n^3)$), exceeding 1s at $n=1000$.}
    \end{minipage}
    \hspace{1em}
    \begin{minipage}[t]{0.4\linewidth}
        \centering
        \includegraphics[width=0.95\linewidth]{figures/distr_vs_msize}
        \captionof{figure}{Runtime variability (100 runs). Significant outliers appear at larger matrix sizes ($n=1000$) due to OS scheduling and caching effects.}
    \end{minipage}
\end{center}
    
\begin{center} 
    
    \begin{minipage}[c]{0.4\linewidth} 
        \centering
        \includegraphics[width=\linewidth]{figures/res_error_increase} 
    \end{minipage}
    \hspace{1em} 
    \begin{minipage}[c]{0.4\linewidth} 
        \captionof{figure}{Residual Error ($||PA-LU||_F$) scaling. Although error grows ($10^{-16} \to 10^{-12}$), it remains negligible.}
    \end{minipage}

\end{center}

}

\vspace*{20em}
\hspace{245pt}

\end{poster}
\end{document}